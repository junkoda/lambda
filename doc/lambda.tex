\documentclass[a4paper,11pt, fleqn]{article}
\usepackage{amsmath}
\usepackage{amssymb}
\usepackage{hyperref}
\usepackage{amsthm} % proof
\usepackage[T1]{fontenc}
%\usepackage{textcomp}
%\usepackage{currvita}
%\usepackage{ascmac}
%\usepackage{color}
\usepackage{xcolor}

\newtheorem{definition}{Definition}
\newtheorem{lemma}{Lemma}
\newtheorem{proposition}{Proposition}
\newtheorem{formula}{Formula}[subsection]
\newtheorem{fs}{Formula}[section]


% Use Times fonts
\usepackage{mathptmx}
\usepackage[scaled]{helvet}
\renewcommand{\ttdefault}{pcr}
\newcommand{\dbar}{\textit{\dj}}
\newcommand{\Dbar}{\textit{\Dj}}
\usepackage{bm}

\setlength{\oddsidemargin}{0pt}   %%% left margin
\setlength{\textwidth}{159.2mm}
\setlength{\topmargin}{0mm}
\setlength{\headheight}{10mm}
\setlength{\headsep}{10mm}         %%% length between header and test
\setlength{\textheight}{219.2mm}

\setlength{\parindent}{0pt}

%
% Colours
%
\definecolor{Green}{rgb}{0.0, 0.5, 0.0}

\begin{document}

%
% Title page
%
\vspace{0.3 \paperheight}
\begin{center}
  {\Huge \textsc{Lambda}}
\end{center}

%
% Redshift-space distortions
%
\newpage

{\Huge \textbf{\textcolor{Green}{Redshift-space distortions}}}

\begin{equation}
  \bm{s} = \bm{x} + u(\bm{x}) \, \hat{\bm{z}}
\end{equation}

%
% Decomposed power
%
\newpage

{\Huge \textbf{\textcolor{Green}{Decomposed power}}}

\vspace{10mm}

We decompose the redshift-space power spectrum to
%
\begin{equation}
  P^s(\bm{k}) = P_{DD}(\bm{k}) + 2 P_{DU}(\bm{k}) + P_{UU}(\bm{k}).
\end{equation}

\vspace{5mm}
\textsc{Definition}
\vspace{-2mm}
%
\begin{align}
  P_{DD}(\bm{k}) &=
  \int\! d^3 r \, e^{-i \bm{k}\cdot\bm{r}}
  \left\langle
    \delta(\bm{x}) \delta(\bm{y}) e^{-ik_z [u(\bm{x}) - u(\bm{y})]}
    \right\rangle,\\
  %
  P_{DU}(\bm{k}) &=
  \int\! d^3 r \, e^{-i \bm{k}\cdot\bm{r}}
  \left\langle
    \delta(\bm{x}) u'(\bm{y}) e^{-ik_z [u(\bm{x}) - u(\bm{y})]}
    \right\rangle,\\
  %
  P_{UU}(\bm{k}) &=
  \int\! d^3 r \, e^{-i \bm{k}\cdot\bm{r}}
  \left\langle
    u'(\bm{x}) u'(\bm{y}) e^{-ik_z [u(\bm{x}) - u(\bm{y})]}
    \right\rangle,
\end{align}
%
where,
\begin{align}
  &u'(\bm{x}) = - \frac{\partial}{\partial z} u(\bm{x}),\\
  &\bm{r}     = \bm{x} - \bm{y},
\end{align}
%
and $\langle\dots\rangle$ is the ensemble average. The ensemble
averages gives a function of $\bm{r}$ due to the statistical
translational invariance.

\vspace{10mm}
$\blacktriangleright$ \hyperlink{link:decomposed-power-derivation}{Derivation}

%
\newpage
\hypertarget{link:decomposed-power-derivation}{The derivation of power spectrum decomposition}

\vspace{10mm}
The redshift-space density contrast in Fourier space is,
%
\begin{equation}
  \label{eq:deltas}
  \delta_D(\bm{k}) + \hat{\delta}^s(\bm{k}) = \int \! d^3 x \, e^{-i\bm{k}\cdot\bm{x}} [1 + \delta(\bm{x})] e^{-ik_z u(\bm{x})},
\end{equation}
%
using,
%
\begin{equation}
  \bm{s} = \bm{x} + u(\bm{x}) \hat{\bm{z}},
\end{equation}
%
and mass conservation,
%
\begin{equation}
  [1 + \delta(\bm{x})] d^3 x = [1 + \delta^s(\bm{s})] d^3 s.
\end{equation}
%
$\delta_D$ is the Dirac delta function,
%
\begin{equation}
  \delta_D(\bm{k}) = \int\! d^3 x \, e^{-i\bm{k}\cdot\bm{x}},
\end{equation}
and $\hat{\delta}^s(\bm{k})$ is the Fourier transform of the
configuration-space density contrast $\delta^s(\bm{s})$,
%
\begin{equation}
  \hat{\delta}^s(\bm{k}) = \int \! d^3 s \,
                           e^{-i\bm{k}\cdot\bm{s}} \delta^s(\bm{s}).
\end{equation}

\vspace{10mm}
We rewrite equation~(\ref{eq:deltas}) as,
%
\begin{align}
  \hat{\delta}(\bm{k}) &= D(\bm{k}) + U(\bm{k}),\\
  D(\bm{k}) &= \int \! d^3 x \, e^{-i\bm{k}\cdot\bm{x}} \delta(\bm{x})
               e^{-ik_z u(\bm{x})},\\
  U(\bm{k}) &= \int \! d^3 x \, e^{-i\bm{k}\cdot\bm{x}}
                              e^{-ik_z u(\bm{x})}.\\
\end{align}
%
The last equation is equal to,
%
\begin{equation}\begin{split}
  U(\bm{k}) = \int\! d^3 x \, e^{-i \bm{k}\cdot\bm{x}}
              \left( - \frac{\partial u(\bm{x})}{\partial z} \right)
              e^{-ik_z u(\bm{x})},
\end{split}\end{equation}
applying integration by parts for,
\begin{equation}
  e^{-i \bm{k}\cdot\bm{x}} = \frac{1}{-ik_z}
  \frac{\partial}{\partial z} e^{-i \bm{k}\cdot\bm{x}}.
\end{equation}

\vspace{10mm}
Power spectra are defined as usual,
\begin{align}
  \left\langle \delta^s(\bm{k}) \delta^s(\bm{k}')^* \right\rangle
    = (2\pi)^3 \delta_D(\bm{k} - \bm{k}') P^s(\bm{k})\\
  \left\langle D(\bm{k}) D(\bm{k}')^* \right\rangle
    = (2\pi)^3 \delta_D(\bm{k} - \bm{k}') P_{DD}(\bm{k})\\
  \left\langle D(\bm{k}) U(\bm{k}')^* \right\rangle
    = (2\pi)^3 \delta_D(\bm{k} - \bm{k}') P_{DU}(\bm{k})\\
  \left\langle U(\bm{k}) U(\bm{k}')^* \right\rangle
    = (2\pi)^3 \delta_D(\bm{k} - \bm{k}') P_{UU}(\bm{k})
\end{align}

\vspace{10mm}
The reality of $P_{DU}$,
%
\begin{equation}
  P_{DU}(\bm{k}) = \left\langle D(\bm{k}) U(\bm{k})^* \right\rangle =
  \left\langle D(\bm{k})^* U(\bm{k}) \right\rangle
\end{equation}

follows from the statistical parity invariance. 

%
% Parity of power spectrum
%
\newpage

{\Huge \textbf{\textcolor{Green}{Parity of power spectra}}}
\vspace{10mm}

Let $P_{ab}(\bm{k})$ be a cross-power spectrum of two real-valued
fields, $a(\bm{x})$ and $b(\bm{x})$,
%
\begin{equation}
  \left\langle \hat{a}(\bm{k}) \hat{b}(\bm{k}') \right\rangle
  = (2\pi)^3 \delta_D(\bm{k} - \bm{k}') P_{ab}(\bm{k}).
\end{equation}

% Lemma 1
\vspace{5mm}
\begin{lemma}
  The cross-power spectrum $P_{ab}$ is an even function,
  \begin{equation}
    P_{ab}(-\bm{k}) = P_{ab}(\bm{k}),
  \end{equation}
  if the product $ab$ is party even,
  \begin{equation}
    a(\bm{x}) b(\bm{x}) \mapsto a(-\bm{x}) b(-\bm{x}),
  \end{equation}
  under parity transformation $\bm{x} \mapsto -\bm{x}$. Similarly, the
  cross power spectrum is an odd function if $ab$ is partity odd.
\end{lemma}

\begin{proof}
  The statistical parity invariance means that the ensemble average does not
  abc
\end{proof}

% Lemma 2
\vspace{5mm}
\begin{lemma}
  The cross-power spectrum $P_{ab}$ is real if it is an even function,
  and it is a pure-imaginary if it is an odd function, respevtively.
\end{lemma}

\begin{proof}
  From the assumption that $a(\bm{x})$ and $b(\bm{x})$ are real-valued
  function, their Fourier transform satify the reality condition,
  %
  \begin{equation}
    \hat{a}(-\bm{k}) = \hat{a}(\bm{k})^*, \,\,
    \hat{b}(-\bm{k}) = \hat{b}(\bm{k})^*.
  \end{equation}
  %
  This immediatly shows,
  %
  \begin{equation}
    \left\langle \hat{a}(-\bm{k}) \hat{b}(-\bm{k})^* \right\rangle
    = \left\langle \hat{a}(\bm{k})^* \hat{b}(\bm{k}) \right\rangle,
  \end{equation}
  %
  and,
  %
  \begin{equation}
    P_{ab}(-\bm{k}) = P_{ab}(\bm{k})^*.
  \end{equation}
  %
  Therefore, an even power spectrun is real,
  \begin{equation}
    P_{ab}(\bm{k}) = P_{ab}(-\bm{k}) = P_{ab}(\bm{k})^*,
  \end{equation}
  and an odd power spectrum is pure imaginary,
  \begin{equation}
    P_{ab}(\bm{k}) = -P_{ab}(-\bm{k}) = -P_{ab}(\bm{k})^*.
  \end{equation}
\end{proof}


The statistical paraity invariance shows that a parity-even power
spectrum is real and a parity-odd power spectrum is pure
imaginary. For example, $P_{DU}$ is real and density-velocity cross power,
%
\begin{equation}
  \left\langle \delta(\bm{k}) u(\bm{k})^* \right\rangle
  = (2\pi)^3 \delta_D(\bm{k} - \bm{k}') P_{\delta u}(\bm{k})
\end{equation}
%
is pure imaginary.

Under the parity transformation,
%
\begin{equation}
  \bm{x} \mapsto -\bm{x},
\end{equation}
scalars are parity even,
%
\begin{equation}
  \delta(\bm{x}) \mapsto \delta(-\bm{x})
\end{equation}
%
and vectors are parity odd,
%
\begin{align}
  &u(\bm{x}) \mapsto -u(-\bm{x}),\\
  &\bm{k} \mapsto - \bm{k}.
\end{align}
%

A cross power spectrum of two real-valued fields,
$a(\bm{x})$ and $b(\bm{x})$,
%
\begin{equation}
  \left\langle \hat{a}(\bm{k}) \hat{b}(\bm{k}')^* \right\rangle
  = (2\pi)^3 \delta_D(\bm{k} - \bm{k}') P_{ab}(\bm{k}),
\end{equation}
%
is a real-valued function if it is parity even, $P_{ab} \mapsto P_{ab}$, and
pure-imaginary function if it is parity odd, $P_{ab} \mapsto -P_{ab}$, respectively.

The reality condition of the Fourier-transformed field,
shows
\begin{equation}
  P_{ab}(-\bm{k}) = P_{ab}(\bm{k})^*
\end{equation}

  
Since the Universe is statistically invariant under parity transformation,



\end{document}
