\documentclass[a4paper,11pt, fleqn]{article}
\usepackage{amsmath}
\usepackage{amssymb}
\usepackage[pdfborder={0 0 0}]{hyperref}
\usepackage{amsthm} % proof
\usepackage[T1]{fontenc}
\usepackage{bm}
\usepackage{xcolor}
\usepackage{sectsty} % change section colours
\usepackage{lipsum}
\usepackage{titlesec}
\usepackage{mathrsfs}

\newtheorem{definition}{Definition}
\newtheorem{lemma}{Lemma}
\newtheorem{proposition}{Proposition}
\newtheorem{formula}{Formula}[subsection]
\newtheorem{fs}{Formula}[section]


% Use Times fonts
\usepackage{newtxtext, newtxmath}
\usepackage[scaled]{helvet}

\newcommand{\dbar}{\textit{\dj}}
\newcommand{\Dbar}{\textit{\Dj}}


\setlength{\oddsidemargin}{0pt}   %%% left margin
\setlength{\textwidth}{159.2mm}
\setlength{\topmargin}{0mm}
\setlength{\headheight}{10mm}
\setlength{\headsep}{10mm}         %%% length between header and test
\setlength{\textheight}{219.2mm}

\setlength{\parindent}{0pt}


%
% Colours
%
\definecolor{Green}{rgb}{0.0, 0.5, 0.0}
\definecolor{Blue}{rgb}{0.16, 0.32, 0.75}
\definecolor{Red}{rgb}{0.82, 0.1, 0.26}


%
% Appearance difinitions
%
%\titlespacing*{\section}{0pt}{5.5ex plus 1ex minus .2ex}{4.3ex plus .2ex}
\titlespacing\section{0pt}{0pt}{24pt}

\title{lambda}
\author{Jun Koda}
\date{}

\begin{document}
%
% Title page
%
\vspace{0.3 \paperheight}

\begin{center}
  {\Huge \textsc{Lambda}}
\end{center}

\tableofcontents

\sectionfont{\sffamily\Huge\color{Green}\selectfont}
\subsectionfont{\sffamily\color{Green}\selectfont}
\subsubsectionfont{\sffamily\color{Green}\selectfont}
\paragraphfont{\sffamily\color{Green}\selectfont}

%
% Redshift-space distortions
%
\newpage

\section{Redshift-space distortions}
%
\begin{equation}
  \bm{s} = \bm{x} + u(\bm{x}) \, \hat{\bm{z}}
\end{equation}

$u(\bm{x})$ is the displacement field with plane-parallel along the $z$ axis.
\begin{equation}
  u(\bm{x}) = \frac{1}{aH} v_z(\bm{x}),
\end{equation}
where $\bm{v}(\bm{x})$ is the peculiar velocity.

\vspace{5mm}

Mass conservation,
%
\begin{equation}
  [1 + \delta(\bm{x})] d^3 x = [1 + \delta^s(\bm{s})] d^3 s,
\end{equation}
%
gives a formula for redshift-space density contrast,
%
\begin{equation}
  \label{eq:delta-s}
  \delta_D(\bm{k}) + \hat{\delta}^s(\bm{k})
  = \int \! d^3 x \, e^{-i\bm{k}\cdot\bm{x}} [1 + \delta(\bm{x})]
    e^{-ik_z u(\bm{x})},
\end{equation}
%
and redshift-space power spectrum,
%
\begin{equation}
  P^s(\bm{k}) = \int \! d^3 r \, e^{-i\bm{k}\cdot\bm{r}}
  \Big\langle
      [1 + \delta(\bm{x})][1 + \delta(\bm{y})]
      e^{-ik_z [u(\bm{x}) - u(\bm{y})]}\Big\rangle.
\end{equation}
$\delta_D$ is the Dirac delta function.


\newpage
\subsection{Linear RSD}

Expanding equation~(\ref{eq:delta-s}) to the linear order in $u$ gives,
%
\begin{align}
  \delta^s(\bm{k}) &= \hat{\delta}(\bm{k}) - ik_z \hat{u}(\bm{k})\\
   &= \hat{\delta}(\bm{k}) + f\mu^2 \hat{\theta}(\bm{k})
\end{align}
%
We call this approximation as the \textit{linear RSD}, which is valid
when higher orders of $k_z u(\bm{x})$ are negligible, but do not
assume that $u(\bm{k})$ or $\hat{\theta}(\bm{k})$ are linear Gaussian
fields; $\delta$ and $\theta$ can be non-linear fields.\\

The \textit{linear} RSD power spectrum is,
%
\begin{equation}
  P^s(\bm{k}) = P_{\delta\delta}(k) +
  2f\mu^2 P_{\delta \theta}(\bm{k}) + f^2\mu^4 P_{\theta\theta}(\bm{k}).
\end{equation}


%
% Decomposed power
%
\newpage
\section{Decomposed power}

We decompose the redshift-space power spectrum as,
%
\begin{equation}
  P^s(\bm{k}) = P_{DD}(\bm{k}) + 2 P_{DU}(\bm{k}) + P_{UU}(\bm{k}).
\end{equation}

\vspace{5mm}
\textsc{Definition}
\vspace{-2mm}
%
\begin{align}
  P_{DD}(\bm{k}) &=
  \int\! d^3 r \, e^{-i \bm{k}\cdot\bm{r}}
  \left\langle
    \delta(\bm{x}) \delta(\bm{y}) e^{-ik_z [u(\bm{x}) - u(\bm{y})]}
    \right\rangle,\\
  %
  P_{DU}(\bm{k}) &=
  \int\! d^3 r \, e^{-i \bm{k}\cdot\bm{r}}
  \left\langle
    \delta(\bm{x}) u'(\bm{y}) e^{-ik_z [u(\bm{x}) - u(\bm{y})]}
    \right\rangle,\\
  %
  P_{UU}(\bm{k}) &=
  \int\! d^3 r \, e^{-i \bm{k}\cdot\bm{r}}
  \left\langle
    u'(\bm{x}) u'(\bm{y}) e^{-ik_z [u(\bm{x}) - u(\bm{y})]}
    \right\rangle,
\end{align}
%
where,
\begin{align}
  &u'(\bm{x}) = - \frac{\partial}{\partial z} u(\bm{x}),\\
  &\bm{r}     = \bm{x} - \bm{y},
\end{align}
%
and $\langle\dots\rangle$ is the ensemble average. The ensemble
averages gives a function of $\bm{r}$ due to the statistical
translational invariance.

\vspace{10mm}
$\blacktriangleright$ \hyperlink{sec:decomposed-power-derivation}{Derivation}

%
\newpage
\subsection{Derivation of power-spectrum decomposition}
\label{sec:decomposed-power-derivation}\quad\\

The redshift-space density contrast in Fourier space is,
%
\begin{equation}
  \delta_D(\bm{k}) + \hat{\delta}^s(\bm{k}) = \int \! d^3 x \, e^{-i\bm{k}\cdot\bm{x}} [1 + \delta(\bm{x})] e^{-ik_z u(\bm{x})},
\end{equation}
%
using,
%
\begin{equation}
  \bm{s} = \bm{x} + u(\bm{x}) \hat{\bm{z}},
\end{equation}
%
and mass conservation,
%
\begin{equation}
  [1 + \delta(\bm{x})] d^3 x = [1 + \delta^s(\bm{s})] d^3 s.
\end{equation}
%
$\delta_D$ is the Dirac delta function,
%
\begin{equation}
  \delta_D(\bm{k}) = \int\! d^3 x \, e^{-i\bm{k}\cdot\bm{x}},
\end{equation}
and $\hat{\delta}^s(\bm{k})$ is the Fourier transform of the
configuration-space density contrast $\delta^s(\bm{s})$,
%
\begin{equation}
  \hat{\delta}^s(\bm{k}) = \int \! d^3 s \,
                           e^{-i\bm{k}\cdot\bm{s}} \delta^s(\bm{s}).
\end{equation}

\vspace{10mm}
We rewrite equation~(\ref{eq:delta-s}) as,
%
\begin{align}
  \hat{\delta}(\bm{k}) &= D(\bm{k}) + U(\bm{k}),\\
  D(\bm{k}) &= \int \! d^3 x \, e^{-i\bm{k}\cdot\bm{x}} \delta(\bm{x})
                                e^{-ik_z u(\bm{x})},\\
  \label{eq:U}
  U(\bm{k}) &= \int \! d^3 x \, e^{-i\bm{k}\cdot\bm{x}}
                              \left[ e^{-ik_z u(\bm{x})} - 1 \right].\\
\end{align}
%
The last equation is equal to,
%
\begin{equation}\begin{split}
  \label{eq:Uprime}
  U(\bm{k}) = \int\! d^3 x \, e^{-i \bm{k}\cdot\bm{x}}
              \left( - \frac{\partial u(\bm{x})}{\partial z} \right)
              e^{-ik_z u(\bm{x})},
\end{split}\end{equation}
applying integration by parts to,
\begin{equation}
  e^{-i \bm{k}\cdot\bm{x}} = \frac{1}{-ik_z}
  \frac{\partial}{\partial z} e^{-i \bm{k}\cdot\bm{x}}.
\end{equation}

\vspace{10mm}
Power spectra are defined as usual,
\begin{align}
  \left\langle \delta^s(\bm{k}) \delta^s(\bm{k}')^* \right\rangle
    = (2\pi)^3 \delta_D(\bm{k} - \bm{k}') P^s(\bm{k})\\
  \left\langle D(\bm{k}) D(\bm{k}')^* \right\rangle
    = (2\pi)^3 \delta_D(\bm{k} - \bm{k}') P_{DD}(\bm{k})\\
  \left\langle D(\bm{k}) U(\bm{k}')^* \right\rangle
    = (2\pi)^3 \delta_D(\bm{k} - \bm{k}') P_{DU}(\bm{k})\\
  \left\langle U(\bm{k}) U(\bm{k}')^* \right\rangle
    = (2\pi)^3 \delta_D(\bm{k} - \bm{k}') P_{UU}(\bm{k})
\end{align}

\vspace{10mm}
The reality of the covariance,
%
\begin{equation}
  \left\langle D(\bm{k}) U(\bm{k}')^* \right\rangle =
  \left\langle D(\bm{k})^* U(\bm{k}') \right\rangle =
  (2\pi)^3 \delta_D(\bm{k} - \bm{k}') P_{DU}(\bm{k})
\end{equation}

follows from the statistical parity invariance. 

\textsc{Proposition}
%
\begin{align}
  \left\langle U(\bm{k}) \right\rangle = 0,\\
  \left\langle D(\bm{k}) \right\rangle = 0
\end{align}


\newpage
\subsection{Power spectrum decomposition in simulation}
\subsubsection{DU fields}

Note that $U(\bm{k})$ field (equation~\ref{eq:U}) is a redshift-space
density field (equation~\ref{eq:delta-s}) with $\delta(\bm{x}) = 0$.
%
\begin{enumerate}
  \item Distribute uniform random particles in a simulation box;
  \item displace the randoms with $u(\bm{x})$. The Fourier transform
    of the density contrast gives the $U(\bm{k})$;
  \item the displaced data particles give $\delta^s(\bm{k})$. The
    residual is,
    \begin{equation}
      D(\bm{k}) = \delta^s(\bm{k}) - U(\bm{k});
    \end{equation}
\end{enumerate}

\vspace{2mm}
\subsubsection{DU power spectra}

The shot-noise-subtracted power spectra are,
%
\begin{align}
  P_{DD}(k, \mu) &= V^{-1} \left\langle D(\bm{k}) D(\bm{k})^* \right\rangle
                  - \bar{n}^{-1}_\mathrm{data} - \bar{n}^{-1}_\mathrm{rand},\\
  P_{DU}(k, \mu) &= V^{-1} \, \mathrm{Re}
                   \left\langle D(\bm{k}) U(\bm{k})^* \right\rangle
                   + \bar{n}^{-1}_\mathrm{rand},\\
  P_{UU}(k, \mu) &= V^{-1} \left\langle U(\bm{k}) U(\bm{k})^* \right\rangle
                  - \bar{n}^{-1}_\mathrm{rand},
\end{align}
where $\bar{n}_\mathrm{data}$ and $\bar{n}_\mathrm{rand}$ are the
number densities of the data and random particles, respectively.

\vspace{5mm}
\paragraph{Shot noise}

The auto-power spectrum of discrete points contains the usual shot
noise,
%
\begin{align}
  V^{-1} \left\langle \delta^s(\bm{k}) \delta^s(\bm{k})^* \right\rangle
  &\sim \bar{n}_\mathrm{data}^{-1},\\
  V^{-1} \left\langle U(\bm{k}) U(\bm{k})^* \right\rangle
  &\sim \bar{n}_\mathrm{rand}^{-1}.
\end{align}
%
The cross power of data and randoms do not have a shot noise in
average because the they are at different positions,
\begin{equation}
  V^{-1} \left\langle \delta^s(\bm{k}) U(k)^* \right\rangle \sim 0.
\end{equation}
%
Therefore, the shot noise in $P_{DU}$ is,
%
\begin{equation}
  V^{-1} \left\langle D(\bm{k}) U(\bm{k})^* \right\rangle
  = V^{-1} \left\langle \delta^s(\bm{k}) U(\bm{k})^* \right\rangle
    - V^{-1} \left\langle U(\bm{k}) U(\bm{k})^* \right\rangle
  \sim -\bar{n}_\mathrm{rand}^{-1}
\end{equation}
%

The shot noise in $P_{DD}$ is,
\begin{equation}
  V^{-1} \left\langle [\delta^s(\bm{k}) - U(\bm{k})]
  [\delta^s(\bm{k}) - U(\bm{k})]^* \right\rangle
  \sim \bar{n}_\mathrm{data}^{-1} + \bar{n}_\mathrm{rand}^{-1}.
\end{equation}



\paragraph{Velocity field interpolation}

Displacing the random particles require the displacement field at
random positions, while the displacement field is provided only at
data particle (halo) positions. We use a piece-wise-constant
interpolation in Voronoi cells, which means we use the velocity of the
nearest data particle. Another common interpolation is a
piece-wise-linear interpolation with the Delaunay tessellation.


%
% Taruya model
%
\newpage
\section{Taruya model}

\vspace{5mm}

Taruya model in the decomposed form is,\vspace{-2mm}
\begin{align}
  P_{DD}(k, \mu) &= b^2 \left[
    P_{mm}(k) + A_{DD}(k, \mu) + B_{DD}(k, \mu)
    \right] e^{- (k\mu\sigma_v)^2},\\
  %
  P_{DU}(k, \mu) &= b \left[
    f\mu^2 P_{m\theta}(k) + A_{DU}(k, \mu) + B_{DU}(k, \mu)
    \right] e^{- (k\mu\sigma_v)^2},\\
  %
  P_{UU}(k, \mu) &= \left[
    f^2 \mu^4 P_{\theta\theta}(k) + A_{UU}(k, \mu) + B_{UU}(k, \mu)
    \right] e^{- (k\mu\sigma_v)^2},   
\end{align}
where $b$ is the linear bias and $m$ stands for matter.\\


\textbf{$A$ terms}\vspace{-2mm}
\begin{align}
  A_{DD}(k, \mu) &= f\mu^2 A_{11}(k)\\
  A_{DU}(k, \mu) &= \frac{1}{2} f^2 \mu^2 \left[
                   A_{12}(k) + \mu^2 A_{22}(k) \right]\\
  A_{UU}(k, \mu) &= f^3 \mu^4 \left[ A_{23}(k) + \mu^2 A_{33}(k) \right]
\end{align}

\textbf{$B$ terms}\vspace{-2mm}
\begin{align}
  B_{DD}(k, \mu) &= f^2 \mu^2 \left[ B^1_{11}(k) + \mu^2 B^2_{11} \right]\\
  B_{DU}(k, \mu) &= -\frac{1}{2} f^3 \left[
    \mu^2( B^1_{12} + B^1_{21} ) + \mu^4(B^2_{12} + B^2_{21})
    + \mu^6(B^3_{12} + B^3_{21}) \right]\\
  B_{UU}(k, \mu) &= f^4 \left[ \mu^2 B^1_{22} + \mu^4 B^2_{22} + \mu^6
    B^3_{22} + \mu^8 B^4_{22}\right]
\end{align}

%
% Streaming model / Cumulant expansion
%
\newpage
\section{Streaming model}
\vspace{-5mm}

\begin{equation}
  P_{ab}(k, \mu) = \mathring{P}_{ab}(k, \mu) \exp \left[
    \frac{A_{ab}}{\mathring{P}_{ab}}
    - \frac{1}{2} \left( \frac{A_{ab}}{\mathring{P}_{ab}} \right)^2
    + \frac{B_{ab}}{\mathring{P}_{ab}}
    + \frac{C_{ab}}{\mathring{P}_{ab}}
    \right],
\end{equation}
%
where $a$, $b$ takes $D$ or $U$.\\

$\mathring{P}_{ab}(\bm{k})$ are the linear RSD,
%
\begin{align}
  &\mathring{P}_{DD}(k) = P_{\delta\delta}(k) = b^2 P_{mm}(k),\\
  &\mathring{P}_{DU}(k, \mu) = P_{\delta u'}(k, \mu) = b f\mu^2 P_{m\theta}(k),\\
  &\mathring{P}_{UU}(k, \mu) = P_{u'u'}(k, \mu) = f^2\mu^4 P_{\theta\theta}(k),
\end{align}
%
$A_{ab}(k, \mu)$ and $B_{ab}(k, \mu)$ are same as the TNS model, and,
%
\begin{equation}
  C_{ab}(k, \mu) =
  \frac{1}{2} \int\!\! d^3 r \, e^{-i\bm{k}\cdot\bm{r}}
  \left\langle u'(\bm{x}) u'(\bm{y}) \right\rangle
  \left\langle (-ik_z \Delta u )^2 \right\rangle.
\end{equation}

%
\begin{equation}
  \sigma_{ab}^2(k, \mu) = \frac{1}{\mathring{P}_{ab}(k, \mu)}
  \int \!\! d^3 r \, e^{-i\bm{k}\cdot\bm{r}} 
  \mathring{\xi}_{ab}(\bm{r}) \xi_{uu}(\bm{r}),
\end{equation}
%
where $\xi$'s are correlation functions,
%
\begin{align}
  \mathring{\xi}_{DD}(\bm{r}) &= \xi_{\delta\delta}(\bm{r}) = \left\langle
  \delta(\bm{x}) \delta(\bm{y}) \right\rangle\\
  \mathring{\xi}_{DU}(\bm{r}) &= \xi_{\delta u'}(\bm{r}) = \left\langle
  \delta(\bm{x}) u'(\bm{y}) \right\rangle\\
  \mathring{\xi}_{UU}(\bm{r}) &= \xi_{u'u'}(\bm{r}) = \left\langle
  u'(\bm{x}) u'(\bm{y}) \right\rangle\\
  \xi_{uu}(\bm{r}) &= \left\langle
  u(\bm{x}) u(\bm{y}) \right\rangle\\
\end{align}
  


\newpage
\subsection{Derivation of the streaming model}

%\vspace{5mm}

We define generating functions, $\mathcal{M}_{ab}$, and apply the cumulant
expansion,
\begin{align}
  \mathcal{M}_{ab}(\bm{k}, J) &\equiv
  \int \!\! d^3 r \, e^{-i\bm{k}\cdot\bm{r}} \left\langle
  \phi_a(\bm{x}) \phi_b(\bm{y}) e^{-iJ\Delta u} \right\rangle\\
  %
  &= \exp\left[
    \mathscr{C}_{ab}^{[0]}(\bm{k}) +
    \mathscr{C}_{ab}^{[1]}(\bm{k}) J +
    \frac{1}{2} \mathscr{C}_{ab}^{[2]}(\bm{k}) J^2 + \cdots
    \right],
\end{align}
where $a$,$b$ takes $D$ or $U$, $\Delta u \equiv u(\bm{x}) - u(\bm{y})$, and,
\begin{align}
  \phi_D(\bm{x}) &\equiv \delta(\bm{x}),\\
  \phi_U(\bm{x}) &\equiv u'(\bm{x}).
\end{align}
The parameter $J$ is a free real number, which reproduces the decomposed power when it is equal to $k_z$,
\begin{equation}
  P_{ab}(\bm{k}) = \mathcal{M}_{ab}(\bm{k}, k_z).
\end{equation}
%
The cumulant expansion is a Taylor expansion of $\ln \mathcal{M}_{ab}$ in powers of $J$,
%
\begin{equation}
  \mathscr{C}_{ab}^{[n]} = \frac{d^n}{dJ^n}
  \ln \mathcal{M}_{ab}(\bm{k}, J) \Big|_{J=0}
\end{equation}

\begin{equation}
  \mathscr{C}_{ab}^{[0]}(\bm{k}) =
  \ln \mathcal{M}_{ab}(\bm{k}, 0) \equiv \ln \mathring{P}_{ab}(\bm{k})
\end{equation}

\begin{align}
  \mathscr{C}_{ab}^{[1]}(\bm{k}) &= \frac{1}{\mathring{P}_{ab}(\bm{k})}
  \int\!d^3 r \, e^{-i\bm{k}\cdot\bm{r}} \left\langle
  \phi_a(\bm{x}) \phi_b(\bm{y}) (-i \Delta u) \right\rangle
\end{align}
%
\begin{align}
  \mathscr{C}_{ab}^{[2]}(\bm{k}) &=
  - \left[ \frac{\mathscr{C}_{ab}^{[1]}}{\mathring{P}_{ab}(\bm{k})} \right]^2
  + \int\!d^3 r \, e^{-i\bm{k}\cdot\bm{r}} \left\langle
  \phi_a(\bm{x}) \phi_b(\bm{y}) (-i \Delta u)^2 \right\rangle
\end{align}


The power spectrum using cumulant expansion is,
\begin{equation}
  P_{ab}(\bm{k}) = \mathcal{M}_{ab}(\bm{k}, k_z) = 
  \mathring{P}_{ab}(\bm{k}) \exp\left[
    k_z \mathscr{C}^{[1]}_{ab}
    + \frac{1}{2} k_z^2 \mathscr{C}^{[2]}_{ab} + \cdots
    \right],
\end{equation}
with,
\begin{equation}
  k_z \mathscr{C}^{[1]}_{ab}(\bm{k})
  = \frac{A(\bm{k})}{\mathring{P}_{ab}(\bm{k})},
\end{equation}
and
\begin{equation}
  \frac{1}{2} k^2_z \mathscr{C}^{[2]}_{ab}(\bm{k})
  = -\frac{1}{2} \left[\frac{A_{ab}(\bm{k})}{\mathring{P}_{ab}(\bm{k})}\right]^2
    + \frac{B_{ab}(\bm{k}) + C_{ab}(\bm{k})}{\mathring{P}_{ab}(\bm{k})}
\end{equation}


The TNS $ABC$ terms are defined as,
\begin{align}
  A_{ab}(\bm{k}) &=
    \int\!d^3 r \, e^{-i\bm{k}\cdot\bm{r}} \left\langle
    \phi_a(\bm{x}) \phi_b(\bm{y}) (-i k_z \Delta u) \right\rangle,\\
  B_{ab}(\bm{k}) &=
   \int\!d^3 r \, e^{-i\bm{k}\cdot\bm{r}}
    \left\langle \bar{\phi}_a(\bm{x})
                  (-ik_z \Delta \bar{u}) \right\rangle
    \left\langle \bar{\phi}_b(\bm{y})
                  (-ik_z \Delta \bar{u}) \right\rangle,\\
  C_{ab}(\bm{k}) &=
    \frac{1}{2} \int\!\! d^3 r \, e^{-i\bm{k}\cdot\bm{r}}
    \left\langle u'(\bm{x}) u'(\bm{y}) \right\rangle
    \left\langle (-ik_z \Delta u )^2 \right\rangle.
\end{align}

The Wick theorem for Gaussian random variables is used for the
4th-order statistics is split to $B$ and $C$,
\begin{equation}
  \left\langle
  \phi_a(\bm{x}) \phi_b(\bm{y}) (\Delta u)^2 \right\rangle
  \approx
  2 \left\langle \bar{\phi}_a(\bm{x})
                  \Delta \bar{u} \right\rangle
    \left\langle \bar{\phi}_b(\bm{y})
                  \Delta \bar{u} \right\rangle
  + \left\langle \bar{\phi}_a(\bm{x})
                 \bar{\phi}_a(\bm{y}) \right\rangle
  \left\langle (\Delta \bar{u})^2 \right\rangle.
\end{equation}

\begin{equation}
  \left\langle (\Delta u)^2 \right\rangle
  = \sigma_v^2 - \xi_{uu}(\bm{r})
\end{equation}
where,
$\sigma_v$ is the rms of the displace field,
%
\begin{equation}
  \sigma_v^2 = \left\langle u(\bm{x})^2 \right\rangle = 
  \frac{f^2}{6\pi^2} \int_0^\infty \!\! P_{\theta\theta}(k) dk.
\end{equation}
%

%
% Numerical integration (xi)
%
\clearpage
\subsection{Numerical integration of the streaming model}

\begin{equation}
  \xi_\ell(r) \equiv \frac{i^\ell}{2\pi^2}
  \int_0^\infty \! k^2 dk \, j_\ell(kr) \bar{P}(k)
\end{equation}

\subsubsection{Integrating correlation functions}

%
% DD
%
\paragraph{DD}

\begin{align}
  \sigma^2_{DD}(k, \mu) &=
  \sigma_v^2 - \sigma^2_{DD(0)}(k)
             - \sigma^2_{DD(2)}(k) \mathcal{P}_2(\mu)\\
%
  \sigma^2_{DD(0)}(k) &= 
  \frac{4\pi}{\bar{P}_{\delta\delta}(k)} \int_0^\infty \! r^2 dr
  j_0(kr)\,
  \bar{\xi}_{\delta\delta}(r)
  \bar{\xi}_{uu(0)}(r)\\
%
  \sigma^2_{DD(2)}(k) &=
  - \frac{4\pi}{\bar{P}_{\delta\delta}(k)} \int_0^\infty \! r^2 dr
  j_2(kr) \,\bar{\xi}_{\delta\delta}(r) \bar{\xi}_{uu(2)}(r)
\end{align}


%
% DU
%
\paragraph{DU}
%
\begin{equation}
  \sigma^2_{DU}(k, \mu) = \frac{1}{f\mu^2 \bar{P}(k)}
  \left[
    \sigma^2_{DU(0)}(k) + \sigma^2_{DU(2)}(k) \mathcal{P}_2(\mu)
    + \sigma^2_{DU(4)}(k) \mathcal{P}_4(\mu)
    \right]
\end{equation}


\begin{align}
  \sigma^2_{DU(0)}(k) &= \frac{4\pi}{3 \mu^2\bar{P}(k)}
    \int\! r^2 dr  j_0(kr) \left[
      \xi_0(r) \xi_{uu(0)}(r) + \frac{2}{5} \xi_2(r) \xi_{uu(2)}(r)
    \right]\\
  %
  \sigma^2_{DU(2)}(k) &= -\frac{4\pi}{3 \mu^2 \bar{P}(k)}
  \int\! r^2 dr j_2(kr) \left[
    \xi_0(r) \xi_{uu(2)}(r) + 2 \xi_2(r) \xi_{uu(0)}(r) + \frac{4}{7} \xi_2(r) \xi_{uu(2)}(r)
    \right] \\
  %
  \sigma^4_{DU(4)}(k) &= \frac{48\pi}{35 \mu^2 \bar{P}(k)} \int\! r^2 dr \,
  \xi_2(r) \xi_{uu(2)}(r) j_4(kr)    
\end{align}

The combination in the exponential $k^2\mu^2 \sigma_{DU}^2(k, \mu)$ is
finite as $\mu \rightarrow 0$.

\begin{equation}
  \bar{\xi}_{\delta u'}(r, \nu)/f
  = \frac{1}{3} \bar{\xi}_0(r) + \frac{2}{3} \bar{\xi}_2(r) \mathcal{P}_2(\nu)
\end{equation}

\begin{equation}
  \mathcal{P}_2(x) \mathcal{P}_2(x)
  = \frac{1}{5} + \frac{2}{7} \mathcal{P}_2(x)
    + \frac{18}{35} \mathcal{P}_4(x)
\end{equation}

%
% UU
%
\clearpage
\paragraph{UU}

\begin{equation} \begin{split}
  \left\langle u'(\bm{x}) u'(\bm{y}) ( \Delta u )^2 \right\rangle
  &=  \left\langle u'(\bm{x}) u'(\bm{y}) \right\rangle
      \left\langle (\Delta u )^2 \right\rangle
      + 2 \left\langle u'(\bm{x}) \Delta u \right\rangle
          \left\langle u'(\bm{y}) \Delta u \right\rangle,\\
  &= 2 \left\langle u'(\bm{x}) u'(\bm{y}) \right\rangle
       \left[ \sigma_v^2 - \xi_{uu}(\bm{r}) \right]
    - 2 \left\langle u'(\bm{x}) u(\bm{y}) \right\rangle
        \left\langle u'(\bm{y}) u(\bm{x}) \right\rangle,\\
  &= 2 \sigma_v^2 \xi_{u'u'}(\bm{r})
     + 2 [\partial_r^2 \xi_{uu}(\bm{r})] \xi_{uu}(\bm{r}) 
     + 2 [\partial_r \xi_{uu}(\bm{r})]^2,\\
  &= 2 \sigma_v^2 \xi_{u'u'}(\bm{r})
     + \partial_r^2 \xi_{uu}(\bm{r})^2,
\end{split}\end{equation}
where,
\begin{equation}
  \partial_r = \frac{\partial}{\partial r_z}.
\end{equation}

\begin{equation} \begin{split}
  \sigma_{UU}^2(\bm{k})
  &\equiv \frac{1}{2}\frac{1}{f^2\mu^4 \bar{P}(k)}
     \int\!\! d^3r \, e^{-i\bm{k}\cdot\bm{r}}
     \left\langle u'(\bm{x}) u'(\bm{y}) ( -i \Delta u )^2 \right\rangle,\\
  &= \sigma_v^2
     + \frac{1}{2 f^2\mu^4 \bar{P}(k)}
       \int\!\! d^3r \, e^{-i\bm{k}\cdot\bm{r}}
       \partial_r^2 \xi_{uu}(\bm{r})^2,\\
  &= \sigma_v^2 - \frac{k^2}{2 f^2 \mu^2 \bar{P}(k)}
     \int\!\! d^3r \, e^{-i\bm{k}\cdot\bm{r}} \xi_{uu}(\bm{r})^2.
\end{split}\end{equation}

\begin{equation}
  \xi_{uu}(r, \nu)^2
  = \left[\xi_{uu(0)}^2 + \frac{1}{5} \xi_{uu(2)}^2 \right]
    + 2 \xi_{uu(2)} \left[
        \xi_{uu(0)} + \frac{1}{7} \xi_{uu(2)} \right]\mathcal{P}_2(\nu)
    +\frac{18}{35} \xi_{uu(2)}^2 \mathcal{P}_4(\nu)
\end{equation}

%
% Numerical integration (xi)
%
\clearpage
\subsubsection{Integrating the power spectrum}

\paragraph{DD}

\begin{align}
  C_{DD}(k, \mu) &=
    \int\! d^3 r \, e^{-i\bm{k}\cdot\bm{r}}
    \bar{\xi}_{DD}(\bm{r}) \bar{\xi}_{UU}d^3 r\\
  &= f^2 \int \! \dbar^3 q \, \bar{P}(|\bm{k} - \bm{q}|) \mu_q^2 P(q)/q^2
\end{align}

The function $P(|\bm{k} - \bm{q}|)P(q)$ is a function of k, q, and $x
= \bm{\hat{k}}\cdot\bm{q}$, for
\begin{equation}
  |\bm{k} - \bm{q}| = \sqrt{k^2 + q^2 - 2kqx}.
\end{equation}
Using Lemma~\ref{lemma-integral} with,
\begin{equation}
  (q_z/q)^2 = \frac{1}{3} + \frac{2}{3} \mathcal{P}_2(q_z/q),
\end{equation}
we obtain,
\begin{align}
  C_{DD}(k, \mu) &= C_{DD(0)}(k) +
                   C_{DD(2)}(k) \mathcal{P}_2(\mu)\\
  C_{DD(0)} &= \frac{f^2 k}{12\pi^2}
                \int_0^\infty \! dr \, \bar{P}(kr)
                \int_{-1}^1 \! dx \, 
                \bar{P}\left(k\sqrt{1 + r^2 - 2rx}\right)\\
  C_{DD(2)} &= \frac{f^2 k}{6\pi^2}
                \int_0^\infty \! dr \, \bar{P}(kr)
                \int_{-1}^1 \! dx \, \mathcal{P}_2(x)
                \bar{P}\left(k\sqrt{1 + r^2 - 2rx}\right)
\end{align}
where the integration variable is converted to $r \equiv q/k$.

\clearpage
\paragraph{DU}

\begin{align}
  C_{DU}(k, \mu) &= \int \! d^3 r \, e^{-i\bm{k}\cdot\bm{r}}
                    \bar{\xi}_{\delta u'}(\bm{r}) \bar{\xi}_{uu}(\bm{r})\\
                 &= f^3 \int \!\frac{d^3 q}{(2\pi)^3}
                    \frac{(k_z - q_z)^2}{|\bm{k} - \bm{q}|^2} \mu_q^2
              \bar{P}\left(|\bm{k} - \bm{q}|\right) \bar{P}(q)/q^2\\
  &= C_{DU(0)} - 2 C_{DU(1)} + C_{DU(2)},
\end{align}
where,
\begin{align}
  C_{DU(n)}(k, \mu)
    &\equiv \frac{f^3 k_z^{2 - n}}{(2\pi)^3}
              \int_0^\infty\! q^n dq \, \bar{P}(q) \int \!d\Omega_q \,
              (q_z/q)^{n + 2}
              \frac{\bar{P}\left( | \bm{k} - \bm{q} |\right)}{|\bm{k} - \bm{q}|^2},\\
    &= \sum_\ell c_{(n+2)\ell} C_{DU(n\ell)}(k) \mathcal{P}_\ell(\mu),
\end{align}
and,
\begin{align}
  &C_{DU(n\ell)} \equiv
    \frac{f^3 k \mu^{2 - n}}{4\pi^2} \int_0^\infty \! r^n dr \, \bar{P}(kr)
    \int_{-1}^1 \!dx \,
    \mathcal{P}_\ell(x)
    \frac{\bar{P}\left(k \sqrt{1 + r^2 - 2rx}\right)}{1 + r^2 -2rx},\\
    &c_{n\ell} \equiv \frac{2\ell + 1}{2}
                     \int_{-1}^1 x^n \mathcal{P}_\ell(x) dx.
\end{align}

\begin{equation}\begin{split}
  &c_{20} = 1/3,\\
  &c_{20} = 2/3,\\
  &c_{31} = 3/5,\\
  &c_{31} = 2/5,\\
  &c_{40} = 1/5,\\
  &c_{43} = 4/7,\\
  &c_{44} = 8/35.
\end{split}\end{equation}

%
% Exact Gaussian
%
\clearpage
\section{Exact Gaussian RSD}

We assume all fields are Gaussian random fields.

The exact-Gaussian $DU$ power spectra are:
%
\begin{equation}
  P_{ab}(\bm{k}) = \int\! d^3 r \,
  \left[ \xi_{ab} - k_z^2 \xi_{au}\xi_{bu} \right]
  e^{-k_z^2 [ \sigma^2_v - \xi_{uu}(\bm{r}) ]},
\end{equation}
%
for $a,b = U \textrm{ or } D$, where,
\begin{align}
  \xi_{ab}(\bm{r}) &=
    \left\langle \phi_a(\bm{x}) \phi_b(\bm{y}) \right\rangle,\\
  \xi_{au}(\bm{r}) &=
    \left\langle \phi_a(\bm{x}) u(\bm{y})\right\rangle.
\end{align}

\vspace{5mm}
\textsc{Derivation}:

\begin{equation}
  P_{ab}(\bm{k}) = \int\! d^3 r \, e^{-i\bm{k}\cdot\bm{r}} \left\langle
                   \phi_a(\bm{x}) \phi_b(\bm{y}) e^{-ik_z \Delta u} \right\rangle,
\end{equation}
%
where $\Delta u \equiv u(\bm{x}) - u(\bm{y})$ and
$\bm{r} \equiv \bm{x} - \bm{y}$.

Define a generating function,
\begin{align}
  \mathcal{M}(\bm{k}, J_1, J_2)
  &\equiv \left\langle e^{-ik_z \Delta u -i J_1 \phi_a(\bm{x}) -i J_2 \phi_b(\bm{y})}
          \right\rangle,\\
  &= \exp \Big\langle -\frac{1}{2} \left[ k_z \Delta u
            + J_1 \phi_a(\bm{x}) + J_2 \phi_b(\bm{y}) \right]^2 \Big\rangle
\end{align}
%
\begin{align}
  P_{ab}(\bm{k}) &= -\frac{\partial}{\partial J_1} \frac{\partial}{\partial J_2}
  \mathcal{M}(\bm{k}, J_1, J_2) \Big|_{J_1 = J_2 = 0}\\
%
  &= \int\! d^3 r \, e^{-i\bm{k}\cdot\bm{r}}
  \left[
    \left\langle \phi_a(\bm{x}) \phi_b(\bm{y}) \right\rangle
    - k_z^2 \left\langle \phi_a(\bm{x}) \Delta u\right\rangle
      \left\langle \phi_b(\bm{y}) \Delta u\right\rangle    
  \right] e^{-\frac{1}{2} k_z^2 \left\langle (\Delta u)^2 \right\rangle}
\end{align}
%
\begin{align}
  \left\langle \phi_a(\bm{x}) \Delta u \right\rangle
    &= -\left\langle \phi_a(\bm{x}) u(\bm{y}) \right\rangle
     = -\xi_{au}(\bm{r})\\
  \left\langle \phi_b(\bm{y}) \Delta u \right\rangle
    &= \left\langle u(\bm{x}) \phi_b(y) \right\rangle
     = \xi_{bu}(-\bm{r}) = -\xi_{bu}(\bm{r})\\
  \left\langle (\Delta u)^2 \right\rangle
  &= 2 \left[\sigma_v^2 - \xi_{uu}(\bm{r})\right]
\end{align}

%
% DD exact Gaussian
%
\clearpage
\subsection{DD}

\begin{align}
  P_{DD}(\bm{k}) &= \int \!\! d^3 r \, e^{-i\bm{k}\cdot\bm{r}}
  \left[ \xi_{\delta\delta}(r) - k_z^2 \xi_{\delta u}(\bm{r})^2 \right]
  e^{-k_z^2 [\sigma_v^2 - \xi_{uu}(\bm{r})]},\\
  &=
  \sum_{n, \ell} \left[ P_{DD, \delta\delta(n\ell)}(k, \mu)
  + P_{DD, \delta u(n\ell)}(k, \mu) \right] \mathcal{P}_\ell(\mu)
\end{align}

$\delta\delta(n)$

\begin{equation}
  P_{DD, \delta\delta (n\ell)}(k, \mu) =
  4\pi (-i)^\ell e^{-(k\mu\sigma_v)^2}
  \frac{1}{n!} \int_0^\infty \! r^2 dr j_\ell(kr) e^{k^2\mu^2 \xi_{uu(0)}(r)}
  \xi_{\delta\delta}(r)
  \left[ k_z^2 \xi_{uu(2)}(r) \right]^n
  \left\langle \mathcal{P}_\ell | \mathcal{P}_2^n \right\rangle
\end{equation}

\begin{equation}
  \left\langle \mathcal{P}_\ell | \mathcal{P}_2^n \right\rangle
  \equiv
  \frac{2\ell + 1}{2} \int_{-1}^1 \!d\nu \,
  \mathcal{P}_\ell(\nu) \mathcal{P}_2(\nu)^n
\end{equation}

\begin{equation}
  P_{DD, \delta u (n\ell)}(k, \mu) =
  4-\pi (-i)^\ell e^{-(k\mu\sigma)^2}
  \frac{k_z^2}{n!} \int_0^\infty \! r^2 dr j_\ell(kr) e^{k^2 \xi_{uu(0)}(r)}
  \xi_{\delta u (1)}(r)^2
  \left[ k_z^2 \xi_{uu(2)}(r) \right]^n
  \left\langle \mathcal{P}_\ell | \nu^2 \mathcal{P}_2^n\right\rangle
\end{equation}

\clearpage
\subsection{UU}

Starting from,
%
\begin{equation}
  U(\bm{k}) = \int \! e^{-i\bm{k}\cdot\bm{x}} e^{-ik_zu(\bm{x})} d^3 x
  - \delta_D(\bm{k})
\end{equation}
%
makes the equation simpler.

\begin{equation}
  \label{eq:puu-exact}
  P_{UU}(\bm{k}) = \int\!\! d^3 r \, e^{-i\bm{k}\cdot\bm{r}}\left\{
  e^{-k_z^2 \left[ \sigma_v^2 - \xi_{uu}(\bm{r}) \right]} - e^{-k_z^2 \sigma_v^2}
  \right\}.
\end{equation}

Note,
%
\begin{equation}
  \int\! d^3 r \, e^{-i\bm{k}\cdot\bm{r}} e^{-k_z^2 \sigma_v^2}
  = \delta_D(\bm{k}) e^{-k_z^2 \sigma_v^2} = \delta_D(\bm{k}),
\end{equation}
%
which makes the integrand of equation~(\ref{eq:puu-exact}) converge to
zero as $k \rightarrow \infty$.

\clearpage
%
% Basics (appendix)
%
\appendix

\sectionfont{\sffamily\Huge\color{Blue}\selectfont}
\subsectionfont{\sffamily\color{Blue}\selectfont}
\subsubsectionfont{\sffamily\color{Blue}\selectfont}
\paragraphfont{\sffamily\color{Blue}\selectfont}

%
% Two-point statistics
%
\section{Two-point statistics}

\paragraph{Fields}
$\mathtt{a}(\bm{x})$, $\mathtt{b}(\bm{x})$ can be the following,
%
\begin{align}
  \delta(\bm{x}) &\phantom{=\,} \mbox{Density contrast field of matter, halo, or galaxies;}\\
            &\phantom{=\,} \mbox{we use $m$ when $\delta$ is explicitly the matter field}\\
  u(\bm{x}) &= \frac{1}{aH} v_z(\bm{x})\\
  u'(\bm{x}) &= -\frac{\partial}{\partial z} u(\bm{x})\\
  \theta(\bm{x}) &= - \frac{1}{aHf} \bm{\nabla}\cdot \bm{v}(\bm{x})
\end{align}

\vspace{5mm}
The field $\hat{\texttt{a}}(\bm{k})$ is
the Fourier transform of $\texttt{a}(\bm{x})$, 
\begin{equation}
  \hat{\texttt{a}}(\bm{k}) =
  \int\! d^3 x \, e^{-i\bm{k}\cdot\bm{x}} \texttt{a}(\bm{x})
\end{equation}

The $u$ field can be solved algebraically in Fourier space:
%
\begin{align}
  &\hat{u}(\bm{k}) = -\frac{ifk_z}{k^2} \hat{\theta}(\bm{x}),\\
  &\hat{u}'(\bm{k}) = f\mu^2 \hat{\theta}(\bm{k}),
\end{align}
%
assuming the vorticity of the velocity field is negligible,
%
\begin{equation}
  \bm{\nabla} \times \bm{v}(\bm{x}) = 0.
\end{equation}

\clearpage
\subsection{Power spectrum}
\begin{equation}
  \left\langle \hat{\mathtt{a}}(\bm{k}) \hat{\mathtt{b}}(\bm{k}')^* \right\rangle
  = (2\pi)^3 \delta_D(\bm{k}-\bm{k}') P_{\mathtt{ab}}(\bm{k})
\end{equation}

\begin{align}
  &P_{uu}(\bm{k}) = f^2\mu^2 P_{\theta\theta}(k)/k^2,\\
  &P_{\delta u'}(\bm{k}) = f \mu^2 P_{\delta \theta}(k),\\
  &P_{u' u'}(\bm{k}) = f \mu^2 P_{\delta \theta}(k),\\
  &P_{\delta u}(\bm{k}) = \frac{if\mu}{k} P_{\delta \theta}(k), \quad
  P_{u \delta}(\bm{k}) = -\frac{if\mu}{k} P_{\delta \theta}(k).
\end{align}


\clearpage
\subsection{Two-point correlation function}

\begin{equation}
  \xi_{\mathtt{ab}}(\bm{r}) = \left\langle \mathtt{a}(\bm{x}) \mathtt{b}(\bm{y})
                             \right\rangle
\end{equation}
%
where $\bm{r} = \bm{x} - \bm{y}$. The ensemble average is a function
of $\bm{r}$ from statistical translational invariance; there is no
special origin $\bm{x} = 0$ in the Universe.\\

\begin{equation}
  \xi_{\delta\delta}(r) = 4\pi \int_0^\infty \! k^2 dk\,
  j_0(kr) P_{\delta\delta}(k)
\end{equation}

\begin{align}
  \xi_{uu}(r, \nu) &= \xi_{uu(0)}(r) + \xi_{uu(2)}(r) \mathcal{P}_2(\nu)\\
  \xi_{uu(0)} &= \frac{f^2}{6 \pi^2} \int_0^\infty \!dk\,
                   j_0(kr) P_{\theta\theta}(k)\\
  \xi_{uu(2)} &= -\frac{f^2}{3 \pi^2} \int_0^\infty \! dk\,
                   j_2(kr) P_{\theta\theta}(k)
\end{align}

\begin{equation}
  \xi_{\delta u}(r, \nu) = -\frac{f}{2\pi^2}
                            \int_0^\infty\! k dk \, j_1(kr) P_{\delta \theta}(k)
                            \mathcal{P}_1(\nu).
\end{equation}

Symmetry,
\begin{equation}
  \xi_{u \delta}(r, \nu) = \xi_{\delta u}(r, -\nu) = -\xi_{\delta u}(r, \nu).
\end{equation}

\clearpage
\subsubsection{Linear correlation functions}
\textsc{Define},
%
\begin{equation}
  \xi_{\ell}(r)
  \equiv 4\pi (-i)^\ell \int_0^\infty \! k^2 dk \, j_\ell(kr) \bar{P}(k)
\end{equation}


The linear correlation functions are,
\begin{align}
  &\xi_{mm}(r) = \xi_0(r),\\
  &\xi_{\delta u'}(r)/f = \frac{1}{3} \xi_0(r) + \frac{2}{3} \xi_2(r),\\
  &\xi_{u'u'}(r)/f^2 = \frac{1}{5} \xi_0(r) + \frac{4}{7} \xi_2(r)
                      +\frac{8}{35} \xi_4(r).
\end{align}

\begin{align}
  x^2 &= \frac{1}{3} \mathcal{P}_0(x) + \frac{2}{3} \mathcal{P}_2(x)\\
  x^4 &= \frac{1}{5}\mathcal{P}_0(x)
  + \frac{4}{7} \mathcal{P}_2(x)
  + \frac{8}{35} \mathcal{P}_4(x)
\end{align}
  
\clearpage
\begin{equation}
  \mathcal{P}_2(\mu) = \frac{1}{2}\left( 3 \mu^2 - 1 \right)
\end{equation}

\begin{equation}
  \mu^2 = \frac{1}{3} + \frac{2}{3} \mathcal{P}_2(\mu)
\end{equation}


\clearpage
\section{Math}

\subsection{Multipole moments}

Multipole moments of a correlation function $\xi(r,\nu)$ are,
\begin{equation}
  \xi_\ell(r) \equiv \frac{2 \ell + 1}{2} \int_{-1}^1 \xi(r, \nu) d\nu;\quad
  \xi(r, \nu) = \sum_{\ell=0}^\infty \xi_\ell(r) \mathcal{P}_\ell(\nu).
\end{equation}
where $\nu = z/r$.\\

Multipole moments of a power spectrum $P(k, \mu)$ are,
\begin{equation}
  P_\ell(k) \equiv \frac{2 \ell + 1}{2} \int_{-1}^1 P(k, \mu) d\mu;\quad
  P(k, \mu) = \sum_{\ell=0}^\infty P_\ell(k) \mathcal{P}_\ell(\nu),
\end{equation}
where $\mu = k_z/k$, and $\mathcal{P}_\ell$ is the \hyperref[sec:legendre]{Legendre polynomial}.\\

\vspace{5mm} Multipole moments $\xi_\ell(r, \nu)$ and $P_\ell(k, \nu)$
are related through the spherical-Bessel transform:
%
\begin{align}
  &P_\ell(k) = 4\pi \, (-i)^\ell \!\int_0^\infty \! j_\ell(kr) \xi_\ell(r) r^2 dr
  \label{eq:spherical-bessel-dr}\\
  &\xi_\ell(r) = \frac{i^\ell}{2\pi^2} \int_0^\infty \! j_\ell(kr) P_\ell(k) k^2 dk
  \label{eq:spherical-bessel-dk}
\end{align}
%
where $j_\ell$ is the spherical Bessel function.

\clearpage

\paragraph{Derivation of spherical-Bessel transform} \quad\\
\label{proof:multipole-transform}

\vspace{5mm}

Apply the spherical-harmonic addition theorem,
%
\begin{equation}
  e^{i\bm{k}\cdot\bm{r}} = 4\pi \sum_{\ell=0}^\infty \sum_{m=-\ell}^\ell
  i^\ell j_\ell(kr) Y_\ell^m(\hat{\bm{k}}) Y_\ell^m(\hat{\bm{r}})^*
\end{equation}
%
to the Fourier transform,
%
\begin{align}
  P(k, \mu) = \int \!\! d^3 r \, e^{-i\bm{k}\cdot\bm{r}} \xi(r, \nu),\\
  \xi(r, \nu) = \int \!\! \frac{d^3 k}{(2\pi)^3} \, e^{i\bm{k}\cdot\bm{r}} P(k, \mu).
\end{align}
%
Then,
%
\begin{equation}
  \int_0^{2\pi} \! Y_\ell^m(\theta, \varphi) \, d \varphi = 0
  \mbox{ for $m \neq 0$},
\end{equation}
%
\begin{equation}
  Y_\ell^0(\theta, \varphi) = \sqrt{\frac{2\ell + 1}{4\pi}} P_\ell(\cos\theta),
\end{equation}
%
and the orthogonality of the Legendre polynomials
(equation~\ref{eq:legendre-orthognal}) give
equations~(\ref{eq:spherical-bessel-dr}--\ref{eq:spherical-bessel-dk}).

\clearpage
\subsection{Numerical integral of spherical-Bessel transform}

We consider a numerical integral spherical-Bessel transform,
%
\begin{equation}
  \int_0^\infty x^n j_\ell(kr) F(x)
\end{equation}
%
Integrating an oscilating function numerically using descritization
much longer than the period of the oscilation is anaccurate. Using a
stepping always smaller than the period is computationally
expensive.

We use a piece-wise-linear interpolation,
%
\begin{equation}
  F(x) = a_0 + a_1 x
\end{equation}
%
and a piece-wise analytical integral,
%
\begin{align}
  &\int x^n \sin x dx = \texttt{sin\_integ}(n, x),\\
  &\int x^n \cos x dx = \texttt{cos\_integ}(n, x),
\end{align}
%
specifically,
%
\begin{align}
  &\int \sin x dx = -\cos x,\\
  &\int \cos x dx = \sin x,\\
  &\int x \sin x dx = \sin x - x \cos x,\\
  &\int x \cos x dx = \cos x + x \sin x,\\
  &\int x^2 \sin x dx = x^2 \sin x + 2 x \cos x - 2 \sin x,\\
  &\int x^2 \cos x dx = -x^2 \cos x + 2 x \sin x + 2 \cos x.
\end{align}
%

\vspace{5mm}

\begin{equation}\begin{split}
    &\int_{x1}^{x2} x^2 (a_0 + a_1 x) j_0(x) dx \\
    &=
    a_0 [\texttt{sin\_integ}(1, x_2) - \texttt{sin\_integ}(1, x_1)]
    + a_2 [\texttt{sin\_integ}(2, x_2) - \texttt{sin\_integ}(2, x_2)]
\end{split}\end{equation}

\clearpage
\subsection{Legendre polynomials}
\label{sec:legendre}

\begin{align}
  &\mathcal{P}_0(x) = 1\\
  &\mathcal{P}_1(x) = x\\
  &\mathcal{P}_2(x) = \frac{1}{2} (3x^2 - 1)\\
  &\mathcal{P}_3(x) = ...\\
  &\mathcal{P}_4(x) = ...
\end{align}

\vspace{4mm}

\paragraph{Orthogonality}
  
\begin{equation}
  \label{eq:legendre-orthognal}
  \int_{-1}^1 \mathcal{P}_\ell(x) \mathcal{P}_m(x) \, dx =
    \frac{2}{2\ell + 1} \delta_{\ell m}.
\end{equation}

\clearpage
\subsection{Integration}
\label{sec:integration}

\vspace{5mm}

\begin{lemma}
  \label{lemma-integral}
  \begin{equation}
    \label{eq:lemma-integral}
    \int \! d\Omega_q \mathcal{P}_\ell(q_z/q) F(k, q, x)
    = \mathcal{P}_\ell(\mu)
      \times 2\pi \int_{-1}^1 \! dx \, F(k, q, x) \mathcal{P}_\ell(x)
  \end{equation}
  where $d\Omega_q$ is the angular integral in the spherical coordinate
  of $\bm{q}$,
  \begin{align}
    &d^3 q = q^2 dq d\Omega_q,\\
    &x \equiv \hat{k}\cdot\hat{\bm{q}},\\
    &\mu = k_z/k
  \end{align}
  and $F$ is an arbitrary
  function of $k$, $q$, and $x$.
\end{lemma}

\begin{proof}
Let us choose a spherical coordinate system of $\bm{q}$, $(q, \theta,
\varphi)$ such that the $\theta = 0$ direction is $\hat{\bm{k}}$,
which gives $x = \hat{\bm{k}}\cdot\bm{\bm{q}} = \cos\theta$. We can expand
the function $F(k, q, x)$ with the Legendre polynomials because it
only depends on $\theta$, not on $\varphi$,
\begin{align}
  &F(k, q, x) = \sum_n F_{n}(k, q) \mathcal{P}_n(\cos\theta),\\
  &F_{n}(k, q) = \frac{2n + 1}{2} \int_{-1}^1 \! dx \,\mathcal{P}_n(x) F(k, q, x)
\end{align}
The lemma immediately follows from substituting these equations to the
left-hand side of equation~(\ref{eq:lemma-integral}) and use the
orthogonality of the Legendre polynomials.
\end{proof}


\clearpage
\section{Basic cosmology}

Scale factor,
\begin{equation}
  a = \frac{1}{1 + \mathrm{redshift}}.
\end{equation}

Hubble parameter,
\begin{equation}
  H(a) = H_0 \sqrt{\Omega_m a^{-3} + \Omega_\Lambda}
\end{equation}
for a flat-$\Lambda$CDM cosmology with only matter and the
cosmological constant, $\Omega_m + \Omega_\Lambda = 1$.\\

Hubble constant,
\begin{equation}
  H_0 = 100\,\mathrm{km}\mathrm{s}^{-1} / \,[h^{-1} \mathrm{Mpc}].
\end{equation}


Linear growth factor,
\begin{equation}
  D(a),
\end{equation}
normalised at present, $D(a=1) = 1$.\\

Linear growth rate,
\begin{equation}
  f = \frac{d\ln D(a)}{d\ln a}.
\end{equation}

%
% Notation
%
\clearpage
\section{Symbols}

\begin{center}
  \begin{tabular}{ll}
    Symbols & \\
    \hline
    $j_\ell$ & Spherical Bessel function\\
    $\mathcal{P}_\ell$ & Legendre polynomial\\
    $Y_\ell^m$ & Spherical harmonics\\
    $\delta_D$ & Dirac delta function\\
    $\delta_{ij}$ & Kronecker delta\\
    $\dbar^3 k$ & $\dbar^3 k = d^3 k / (2\pi)^3$\\ 
    \hline
    %
    $a$ & Scale factor, $a=1$ at present\\
    $H$ & Hubble parameter (time dependent)\\
    $D$ & Linear growth factor\\
    $f$ & Linear growth rate $f = d \ln D(a)/d\ln a$\\
    $\delta(\bm{x})$ & Density contrast field of matter, haloes, or galaxies\\
    
    $\theta(\bm{x})$ & Velocity divergence field
    $\theta(\bm{x}) = -\bm{\nabla} \cdot \bm{v}(\bm{x})$\\
    $u(\bm{x})$ & Redshift-space displacement $u(\bm{x}) = v_z(\bm{x})/(aH)$\\
    %
    $\bm{r}$ & Separation vector $\bm{r} = \bm{x} - \bm{y}$\\
    $\nu$   & $\nu = z/r = r_3/|\bm{r}|$\\
    $\Delta u$ & Pairwise RSD displacement $\Delta u = u(\bm{x}) - u(\bm{y})$\\
    $k$ & Magnitude of the wavevector $k = |\bm{k}|$\\
    $\mu$ & $\mu=k_z/k$\\
    \hline
  \end{tabular}
\end{center}

%
%
%


\end{document}
